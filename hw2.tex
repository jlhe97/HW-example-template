ss[12pt]{article}

%\newcommand{\name}{Bradford Bonanno}
%\newcommand{\problemset}{ Homework 1 }

%\pagestyle{headings}
\usepackage[dvips]{graphics,color}
\usepackage{amsfonts}
\usepackage{amssymb}
\usepackage[fleqn]{amsmath}
\usepackage{latexsym}
\usepackage{enumerate}
\usepackage{tcolorbox}
\setlength{\parskip}{1pc}
\setlength{\parindent}{0pt}
\setlength{\topmargin}{-3pc}
\setlength{\textheight}{9.5in}
\setlength{\oddsidemargin}{0pc}
\setlength{\evensidemargin}{0pc}
\setlength{\textwidth}{6.5in}

%% The graphicx package provides the includegraphics command.
\usepackage{graphicx}

\newcommand{\answer}[2]{
    \newpage
        \noindent
        \framebox{
            \vbox{
                Homework \hfill {\bf \problemset}
                %        \hfill \# #1 \\ 
                    \name \hfill \today \\
                    %                Collaborators: #2
            }
        }
    \bigskip

}


\begin{document}
MA463X Homework Assignment 2\\
           Name: Bradford Bonanno, Juan Luis Herrero Estrada\\
           Date: March 27, 2017\\
           --------------------------------------------------------------------------------------------------

           \begin{enumerate}
           \item Suppose we have a data set with five predictors, \(X_1\) = GPA, \(X_2\) = IQ, \(X_3\) = Gender (1 for Female and 0 for Male), $X_4$ = Interaction between GPA and IQ, and $X_5$ = Interaction between GPA and Gender. The response is starting salary after graduation (in thousands of dollars). Suppose we use least squares to fit the model, and get $\hat{\beta_0}$ = 50, $\hat{\beta_1}$ = 20, $\hat{\beta_2}$ = 0.07, $\hat{\beta_3}$ = 35, $\hat{\beta_4}$ = 0.01, $\hat{\beta_5}$ = −10.
           \begin{enumerate}
           \item Which answer is correct, and why?
           \begin{enumerate}
           \item For a fixed value of IQ and GPA, males earn more on average than females.\\

           \item For a fixed value of IQ and GPA, females earn more on average than males \\

           \item For a fixed value of IQ and GPA, males earn more on average than females provided that the GPA is high enough.\\

           \item For a fixed value of IQ and GPA, females earn more on average than males provided that the GPA is high enough.\\

           \end{enumerate} %end of 1a.
           \item Predict the salary of a female with IQ of 110 and a GPA of 4.0. \\ % end of 1b.

           \item True or false: Since the coefficient for the GPA/IQ interaction term is very small, there is very little evidence of an interaction effect. Justify your answer.\\ % end of 1c.

\end{enumerate} % end of 1 (lesson 3.7 #3)

\item Using (3.4), argue that in the case of simple linear regression, the least squares line always passes through the point ($\bar{x}$, $\bar{y}$). \\ % end of end of 2 (lesson 3.7 #6)

    \item In this exercise you will create some simulated data and will fit simple linear regression models to it. Make sure to use set.seed(1) prior to starting part (a) to ensure consistent results.
    \begin{enumerate}
    \item Using the rnorm() function, create a vector, x, containing 100 observations drawn from a N(0,1) distribution. This represents a feature, X. \\



    \item Using the rnorm() function, create a vector, eps, containing 100 observations drawn from a N(0,0.25) distribution i.e. a normal distribution with mean zero and variance 0.25.\\



    \item Using x and eps, generate a vector y according to the model.\\ \begin{center} $Y = -1 + 0.5X+ \epsilon$ \end{center} What is the length of the vector y? What are the values of β0 and β1 in this
    linear model? \\

    %answer goes here

    \item Create a scatterplot displaying the relationship between x and y. Comment on what you observe. \\

    %answer goes here

    \item Fit a least squares linear model to predict y using x. Comment on the model obtained. How do $\hat{\beta_0}$ and $\hat{\beta_1}$ compare to
    $\hat{\beta_0}$ and $\hat{\beta_1}$? \\

    %answer goes here

    \item Display the least squares line on the scatterplot obtained in (d). Draw the population regression line on
    the plot, in a different color. Use the legend() command to create an appropriate legend. \\

    %answer goes here

    \item Now fit a polynomial regression model that predicts y using x and $x^2$. Is
    there evidence that the quadratic term improves the model fit? Explain your
    answer. \\

    %answer goes here

    \item Repeat (a)–(f) after modifying the data generation
    process in such a way that there is less noise in the
    data. The model (3.39) should remain the same. You can do
    this by decreasing the variance of the normal distribution
    used to generate the error term $\epsilon$ in (b).
    Describe your results. \\

    %answer goes here

    \item Repeat (a)–(f) after
    modifying the data
    generation process in such
    a way that there is more
    noise in the data. The
    model (3.39) should remain
    the same. You can do this
    by increasing the variance
    of the normal distribution
    used to generate the error
    term $\epsilon$ in (b).
    Describe your results. \\

    %answer
    %goes here

    \item
    Do
    cross
    validation
    on
    the
    data
    (changed
     by
     Prof.
     Paffenroth)
    \\

    %answer
    %goes
    %here

    \end{enumerate}
    % end
    % of
    % question
    % 3
    % (lesson
            % 3.7
            % #13)

    \item
    This
    problem
    involves
    the
    Boston
    data
    set,
    which
    we
    saw
    in
    the
    lab
    for
    this
    chapter.
    We
    will
    now
    try
    to
    predict
    per
    capita
    crime
    rate
    using
    the
    other
    variables
    in
    this
    data
    set.
    In
    other
    words,
    per
    capita
    crime
    rate
    is
    the
    response,
    and
    the
    other
    variables
    are
    the
    predictors.
    \begin{enumerate}
    \item
    For
    each
    predictor,
    fit
    a
    simple
    linear
    regression
    model
    to
    predict
    the
    response.
    Describe
    your
    results.
    In
    which
    of
    the
    models
    is
    there
    a
    statistically
    significant
    association
    between
    the
    predictor
    and
    the
    response?
    Create
    some
    plots
    to
    back
    up
    your
    assertions.
    \\

    %answer
    %goes
    %here

    \item
    skip
    part
    b
    (changed
     by
     Prof.
     Paffenroth)

    \item
    How
    do
    your
    results
    from
(a)
    compare
    to
    your
    results
    from
    (b)?
    Create
    a
    plot
    displaying
    the
    univariate
    regression
    coefficients
    from
(a)
    on
    the
    x-axis,
    and
    the
    multiple
    regression
    coefficients
    from
(b)
    on
    the
    y-axis.
    That
    is,
    each
    predictor
    is
    displayed
    as
    a
    single
    point
    in
    the
    plot.
    Its
    coefficient
    in
    a
    simple
    linear
    regression
    model
    is
    shown
    on
    the
    x-axis,
    and
    its
    coefficient
    estimate
    in
    the
    multiple
    linear
    regression
    model
    is
    shown
    on
    the
    y-axis.
    \\

    %answer
    %goes
    %here

    \item
    Is
    there
    evidence
    of
    non-linear
    association
    between
    any
    of
    the
    predictors
    and
    the
    response?
    To
    answer
    this
    question,
    for
    each
    predictor
    X,
    fit
    a
    model
    of
    the
    form
    \begin{center} 
    $Y
    =
    \beta_0
    +
    \beta_1X
    +
    \beta_2X^2
    +
    \beta_3X^3
    +
    \epsilon$ 
    \end{center}

    %answer
    %goes
    %here

    \end{enumerate}
    % end
    % of
    % question
    % 4
    % (lession
            % 3.7
            % #15)

    \end{enumerate}
    % end
    % of
    % all
    % the
    % problems.

    \end{document}




